%% french-lite.tex
%% Copyright 2015-2020,2022 Gaël PORTAY <gael.portay@gmail.com>
%
% This work may be distributed and/or modified under the
% conditions of the LaTeX Project Public License, either version 1.3
% of this license or (at your option) any later version.
% The latest version of this license is in
%   http://www.latex-project.org/lppl.txt
% and version 1.3 or later is part of all distributions of LaTeX
% version 2005/12/01 or later.
%
% This work has the LPPL maintenance status `maintained'.
%
% The Current Maintainer of this work is Gaël PORTAY.
%
% This work consists of the files french-lite.tex and profile.png.

\documentclass[11pt,a4paper,sans]{moderncv}

\moderncvstyle{classic}
\moderncvcolor{blue}

\usepackage[scale=0.75]{geometry}
\usepackage[utf8]{inputenc}

\name{Gaël}{PORTAY}
\title{Consultant en Logiciels Libres}
\email{gael.portay@gmail.com}
\homepage{www.portay.io}
\social[linkedin]{gaël-portay-80399360}
\social[github]{gportay}
\photo[64pt][0.4pt]{profile.png}

\begin{document}
\makecvtitle

\section{Expériences Professionnelles}
\cventry{Novembre 2018--Juin 2022}{Développeur Logiciel Senior}{Collabora}{Montréal}{}{
\begin{itemize}
\item Investigation et débogage du noyau Linux
\begin{itemize}
\item Analyse de la diminution de performances de transfert sur le bus SPI (kernelshark)
\item Résolution de deux blocages systèmes du pilote V4L2 de la plateforme i.MX6
\item Analyse d'un blocage système dans l'allocation de mémoire continue (CMA)
\end{itemize}
\item Création d'une distribution dédiée au jeu vidéo basée sur Arch Linux
\begin{itemize}
\item Modification/création de paquets et mise en place du dépôt (pacman, makepkg)
\item Création d'une image et des artefacts de mise à jour (RAUC, casync)
\item Mise en place de la configuration et développement de plugins pour l'installateur multi platforme (calamares)
\item Création de scripts et de conteneurs (bash, Docker)
\item Redistribution des contributions aux logiciels libres (systemd, plymouth, grub...)
\end{itemize}
\item Création de plugins fwupd/LVFS pour la mise à jour de micro logiciels
\begin{itemize}
\item Ajout des hubs USB Genesys Logic et des moniteurs USB-C HP
\item Ajout des souris «~esport~» Steelseries via les connections sans-fil 2.4G et filaire USB
\end{itemize}
\end{itemize}
}
\cventry{Mars 2016--Octobre 2018}{Consultant en Logiciel Libre}{Savoir-Faire Linux}{Montréal}{}{
\begin{itemize}
\item Développement d'un framework d'échange de données entre différents objets connectés (C++ 11, ZeroMQ, Protobuf, ld, gcov)
\item Mise à jour du logiciel embarqué vers une nouvelle version (Yocto, systemd)
\item Personnalisation de l'interface d'un périphérique réseau (OpenWRT, Lua/LuCI) et ajout du support de conteneurisation (docker).
\item Support du paquet QtWebEngine et Raspberry Pi 3 (64-bits) dans Buildroot (Makefile)
\end{itemize}
}
\cventry{Juillet 2010--Septembre 2015}{Ingénieur Linux Embarqué}{Overkiz SAS, groupe Somfy}{Archamps}{}{
\begin{itemize}
\item Coresponsable de la distribution Linux embarqué
\item Intégration d'outils issus de la communauté du logiciel libre
\item Responsable des mises à jour du logiciel embarqué (OPKG, paquets Debian)
\item Mise en place du système de «~Build automatisé~» Yocto (Python, shell)
\item Développement de frameworks et d'applications (C/C++)
\item Développement de modules noyaux et BSP Linux (C, device-tree, Git)
\item Support d'UBI dans at91bootstrap pour sécurisation de mises à jour critiques (C)
\end{itemize}
}

\pagebreak
\section{Stages}
\cventry{2008}{Stage Ingénieur de 3e année}{Freescale Semiconductors}{Toulouse}{}{Développement d'un pilote de charge de batterie Lithium-ion sous Nokia S60 (Symbian OS)}
\cventry{2007}{Stage Ingénieur de 2ème année}{Sagem Monetel}{Valence}{}{Optimisation de l’emprunte mémoire d'une application C embarquée ; 30\% de gain}
\cventry{2004}{Stage Technicien de 2ème année de DUT}{Sagem Monetel}{Valence}{}{Portage d'application C embarquée vers le compilateur libre GNU/GCC ; développement d'un outil de benchmarking}

\section{Éducation}
\cventry{2008--2009}{Master MAE}{IAE}{Grenoble}{\textit{Bac +5}}{Année spéciale de Master en Management des Administrations et des Entreprises}
\cventry{2005--2008}{3I}{Polytech'Grenoble}{Grenoble}{\textit{Bac +5}}{Diplôme d’ingénieur en Informatique Industrielle et Instrumentation}
\cventry{2004--2005}{Licence TIC}{Université de Savoie}{Chambéry}{\textit{Bac +3}}{2e année de Licence en Technologie de l’Information et de la Communication}
\cventry{2002--2004}{DUT ISI}{IUT de Valence}{Valence}{\textit{Bac +2}}{Diplôme universitaire de Technologie en Informatique et Systèmes Industriels}
\cventry{2001--2002}{DEUG SV}{Université de Savoie}{Chambéry}{\textit{Bac +2}}{1ère année de Diplôme d’Etudes Universitaires Générales en Science de la Vie}

\section{Langues}
\cvitemwithcomment{Français}{Langue maternelle}{}
\cvitemwithcomment{Anglais}{Bonnes connaissances}{}

\section{Compétences techniques}
\cvitem{Langages de programmation}{Shell, Makefile, C/C++}
\cvitem{Autres}{Linux Kernel, Git, Cross-compilation, Yocto, Buildroot, Docker}

\section{Références}
\cvitem{Overkiz}{Florent PELLARIN,\linebreak[0]Directeur des Opérations\linebreak[0](\href{mailto:f.pellarin@overkiz.com}{f.pellarin@overkiz.com})}
\cvitem{Savoir-Faire Linux}{Jérôme OUFELLA,\linebreak[0]Vice Président Technologie\linebreak[0](\href{mailto:jerome.oufella@savoirfairelinux.com}{jerome.oufella@savoirfairelinux.com})}
\cvitem{Collabora}{Dave BEVAN,\linebreak[0]Responsable du Personnel d'Ingénierie\linebreak[0](\href{mailto:dave.bevan@collabora.com}{dave.bevan@collabora.com})}

\end{document}
