%% coverletter-sfl.tex
%% Copyright 2015 Gaël PORTAY <gael.portay@gmail.com>
%
% This work may be distributed and/or modified under the
% conditions of the LaTeX Project Public License, either version 1.3
% of this license or (at your option) any later version.
% The latest version of this license is in
%   http://www.latex-project.org/lppl.txt
% and version 1.3 or later is part of all distributions of LaTeX
% version 2005/12/01 or later.
%
% This work has the LPPL maintenance status `maintained'.
%
% The Current Maintainer of this work is Gaël PORTAY.
%
% This work consists of the file coverletter-sfl.tex.

%\documentclass{moderncv}
\documentclass[11pt,a4paper,sans]{moderncv}

\moderncvtheme[blue]{casual}

\usepackage[scale=0.75]{geometry}
\usepackage[utf8]{inputenc}
\usepackage[francais]{babel}

\name{Gaël}{PORTAY}
\title{Curriculum vit\ae}
\address{12 chemin de Mireille}{74200 Thonon-les-bains}{France}
\phone[mobile]{+33 (0)6 63 76 57 11}
\email{gael.portay@gmail.com}
\homepage{www.portay.fr/gael}
\social[linkedin]{gaël-portay}
\social[twitter]{gazoo74}
\social[github]{gazoo74}
\extrainfo{Ingénieur Linux Embarqué}

\recipient{Savoir Faire Linux}{7275 Saint Urbain - Bureau 200\\Montréal, QC, H2R 2Y5\\Canada}
\date{\today}
\opening{Bonjour,}
\closing{Cordialement,}
\enclosure[Ci-joint]{Curriculum Vit\ae{} et portfolio.}

\begin{document}
\makelettertitle

Je vous addresse cette lettre suite à la parution de vos offres d'emploi sur votre \href{https://carrieres.savoirfairelinux.com/}{site Internet} ainsi que sur celui des « \href{http://journeesquebec.gouv.qc.ca/}{\textit{Journées Quebec}} » à Paris. En effet le poste de « \href{https://carrieres.savoirfairelinux.com/#consultant-logiciel-pour-systemes-embarques}{\textit{Consultant logiciel pour systèmes embarqués}} » a particulierement retenu mon attention.

Ces cinq dernières années, j'ai acquis de réelles connaissances dans le domaine de \textit{Linux embarqué}. Au niveau espace utilisateur tout d'abord, en développant des applications et des librairies pour des systèmes embarqués (C/C++, epoll, compilation-croisée...). Puis j'ai approfondi mes compétences en système d'exploitation en développant une distribution \textit{Linux} maison depuis zéro (système d'initialisation, gestionnaire de paquets, gestion dynamique des périphériques...). Plus récemment, j'ai travaillé dans le noyau \textit{Linux} et sur les étapes du démarrage d'un tel système lorsque j'ai développé les \textit{BSPs} de deux nouvelles électronique à base de \textit{SoC} Atmel.

Ces expériences m'ont confrontées plusieurs fois à la communauté du \textit{Logiciel Libre} si bien que je souhaite marquer un tournant dans ma carrière. Aujourd'hui, je cherche à évoluer vers un poste de consutant afin de partager à la fois mes connaissances et mes développements. Vous trouverez de plus amples informations sur mon cursus à travers mon \href{http://portay.fr/journees-quebec/pdf/french.pdf}{\textit{CV}} et sur mes compétences techniques via mon \href{http://portay.fr/journees-quebec/pdf/portfolio-french.pdf}{\textit{portfolio}}, tous deux joint.

Je vous remercie par avance de l'attention que vous porterez à ma candidature. J'espère avoir l'opportunité de vous rencontrer aux Journées Quebec à Paris le 21 et 22 novembre.

\makeletterclosing
\end{document}
